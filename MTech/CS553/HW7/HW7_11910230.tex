\documentclass[12pt]{article}

\usepackage[a4paper, left = 1.5cm, right = 1.5cm, top = 1.5cm, bottom = 1.5cm]{geometry}

\usepackage{amsmath}
\usepackage{amssymb}

\usepackage{pgfplots}
\usepackage{pgfplotstable}
\pgfplotsset{width=15cm,height=7cm,compat=newest}
\usepgfplotslibrary{fillbetween}

\usepackage{mathtools}

\usepackage{enumitem}

\usepackage{graphicx}

\usepackage{diagbox}

\usepackage{karnaugh-map}

\usepackage[T1]{fontenc}

\usepackage{listings}
\usepackage{xcolor}
\usepackage{colortbl}

\usepackage{tabu}

\lstset{
	language = Python,
	backgroundcolor=\color{black!5}, % set backgroundcolor
	breaklines=true,
	postbreak=\mbox{\textcolor{red}{$\hookrightarrow$}\space},
}

\renewcommand{\baselinestretch}{1.5}

\renewcommand\thesection{\arabic{section}.}

\usepackage{tocloft}
\cftsetindents{section}{0em}{2em}
\cftsetindents{subsection}{0em}{2em}
\renewcommand\cfttoctitlefont{\hfill\Large\bfseries}
\renewcommand\cftaftertoctitle{\hfill\mbox{}}
\setcounter{tocdepth}{2}

\begin{document}
\begin{titlepage}
\centering
\vspace*{\fill}
\huge CS553: Cryptography\\
\LARGE Assignment 7: Solutions\\
\Large Rohit Das (11910230)\\\vspace{0.8cm}
\today
\vspace*{\fill}
\end{titlepage}

\section{Mix-Column Transitions}

Python Code (Python3): AES.py
\lstinputlisting[language=Python,showstringspaces=false]{./python_code/AES.py}
\vspace{0.05cm}

\newpage
\section{Integral Distinguisher}

Python Code (Python3): integral$\_$crypt.py
\lstinputlisting[language=Python,showstringspaces=false]{./python_code/integral_crypt.py}
\vspace{0.05cm}
\textbf{Output:}
Row: 1, Col: 1, Key: 1234
\lstinputlisting[showstringspaces=false]{./python_code/integral_output.txt}

\newpage
\section{Fault Tolerance: Sypher004}

Python Code (Python3): Sypher004.py
\lstinputlisting[language=Python,showstringspaces=false]{./python_code/Sypher004.py}
\vspace{0.05cm}
Output:
IV: 1234. Message: 1234abcd.
\lstinputlisting[showstringspaces=false]{./python_code/Sypher004output.txt}
\vspace{0.05cm}

\newpage
\section{Modes}

\subsection{ECB Encryption}
Bash Script File: enc$\_$pic.sh
\lstinputlisting[language=Bash,showstringspaces=false]{./ECB_Pic/enc_pic.sh}
\vspace{0.05cm}

\begin{figure}[h!]

\centering\includegraphics[width = 0.45\linewidth]{./ECB_Pic/IITBhilaiLogo.png}
\centering\includegraphics[width = 0.45\linewidth]{./ECB_Pic/IITBhilaiLogoEnc.png}

\caption{Left: Original IIT Bhilai Logo. Right: Encrypted using aes-256-cbc}

\end{figure}

\subsection{Need for Pre-IV}

In CBC and CFB modes of encryption, being able to predict the IV will lead to leaking the message itself. To make the IV unpredictable, a notion of Pre-IV is used. It can be a nonce generated from a counter, which is then encrypted to create the IV.

\begin{center}

$IV = ENC_k$(PRE-IV)

\end{center}
Other theoretical works suggest using some key k' derived from k that is used for encryption.
\begin{center}

$IV = ENC_{k'}$(PRE-IV)

\end{center}

\begin{large}

\end{large}

\end{document}