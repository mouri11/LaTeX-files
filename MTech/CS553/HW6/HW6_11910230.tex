\documentclass[12pt]{article}

\usepackage[a4paper, left = 1.5cm, right = 1.5cm, top = 1.5cm, bottom = 1.5cm]{geometry}

\usepackage{amsmath}
\usepackage{amssymb}

\usepackage{pgfplots}
\usepackage{pgfplotstable}
\pgfplotsset{width=15cm,height=7cm,compat=newest}
\usepgfplotslibrary{fillbetween}

\usepackage{mathtools}

\usepackage{enumitem}

\usepackage{graphicx}

\usepackage{diagbox}

\usepackage{karnaugh-map}

\usepackage[T1]{fontenc}

\usepackage{listings}
\usepackage{xcolor}
\usepackage{colortbl}

\usepackage{tabu}

\lstset{
	language = Python,
	backgroundcolor=\color{black!5}, % set backgroundcolor
}

\renewcommand{\baselinestretch}{1.5}

\renewcommand\thesection{\arabic{section}.}

\usepackage{tocloft}
\cftsetindents{section}{0em}{2em}
\cftsetindents{subsection}{0em}{2em}
\renewcommand\cfttoctitlefont{\hfill\Large\bfseries}
\renewcommand\cftaftertoctitle{\hfill\mbox{}}
\setcounter{tocdepth}{2}

\begin{document}
\begin{titlepage}
\centering
\vspace*{\fill}
\huge CS553: Cryptography\\
\LARGE Assignment 6: Solutions\\
\Large Rohit Das (11910230)\\\vspace{0.8cm}
\today
\vspace*{\fill}
\end{titlepage}

\section{Linear Approximation Table}

\begin{large}
Python code (Python 3): LAT\textunderscore s-box.py

\lstinputlisting[language = Python, showstringspaces=false]{./python_code/LAT_s-box.py}
\vspace{0.05cm}

\begin{center}

\begin{tabular}{c||*{15}{p{0.5cm}}}

in\textbackslash out & 1 & 2 & 3 & 4 & 5 & 6 & 7 & 8 & 9 & a & b & c & d & e & f\\\hline
1 &  .&-2&-2& .& 4&-2& 2& .& .&-2&-2&-4& .& 2&-2\\
2 &  2& .& 2&-4& 2& .&-2& .&-2& .&-2& .& 2&-4&-2\\
3 & -2& 2& .& 4& 2& 2& .& 4&-2&-2& .& .& 2&-2& .\\
4 & -2&\cellcolor[rgb]{0.96,0.37,0.37}6& .& .&-2&-2& .&-2& .& .&-2&-2& .& .&-2\\
5 & -2& .& 2& .& 2& .&\cellcolor[rgb]{0.96,0.37,0.37}6&-2& .& 2& .& 2& .&-2& .\\
6 &  4& 2& 2& .& .&-2& 2& 2&-2& .& .&-2&-2& .& 4\\
7 &  .& .&-4& .& .& 4& .&-2&-2& 2&-2&-2&-2&-2& 2\\
8 &  .& 2&-2&-2& 2& .& .& 4& 4& 2&-2& 2&-2& .& .\\
9 &  .& .&-4&-2&-2&-2& 2& .& .&-4& .& 2& 2&-2& 2\\
a & -2&-2& .& 2& .&-4&-2& .& 2& 2& .&-2& .&-4& 2\\
b &  2& .&-2& 2& .&-2& .& .&-2& 4&-2& 2& 4& 2& .\\
c & -2& .&-2&-2& .&-2& .& 2&-4& 2& 4& .&-2& .&-2\\
d & -2& 2& .&-2& 4& .&-2&-2& .& .& 2& .& 2& 2& 4\\
e &  .& .& .&-2&-2& 2& 2& 2& 2& 2& 2&-4& 4& .& .\\
f & -4&-2& 2&-2&-2& .& .& 2&-2& .&-4& .& .& 2& 2\\

\end{tabular}

\normalsize Linear Approximation Table
\end{center}
\vspace{0.15cm}
The LAT above is generated for the S-box given below:

\begin{center}

\begin{tabular}{|c|c|c|c|c|c|c|c|c|c|c|c|c|c|c|c|c|}
\hline
$x$ & 0 & 1 & 2 & 3 & 4 & 5 & 6 & 7 & 8 & 9 & a & b & c & d & e & f\\\hline
$S(x)$ & 5 & 4 & d & 1 & 2 & f & 6 & 0 & 8 & c & b & 9 & 7 & e & a & 3\\\hline
\end{tabular}

\end{center}

From the LAT generated, we can see that the characteristics 4 $\xrightarrow{\text{S}}$ 2 and 5 $\xrightarrow{\text{S}}$ 7 will have the highest probability of producing a linear approximation of the above Sbox with p = $\dfrac{1}{2}$ + $\dfrac{6}{16}$.

\end{large}
\vspace{0.5cm}

\section{Bi-directional LC of Sypher00C}

\begin{large}

16-bit key used = (6||f||e||c)\\
Python code (Python 3): LC\textunderscore Sypher00C.py
\lstinputlisting[language = Python, showstringspaces=false]{./python_code/LC_Sypher00C.py}

\begin{center}

\begin{tabular}{*{17}{c|}}
\diagbox{p-c}{$k_3$} & 0 & 1 & 2 & 3 & 4 & 5 & 6 & 7 & 8 & 9 & a & b & c & d & e & f \\\hline
0-d&   0& 1& 0& 1& 1& 0& 0& 0& 1& 0& 1& 1& 0& 1& 0& 1\\\hline
1-b&   1& 1& 0& 1& 1& 0& 1& 0& 1& 0& 1& 0& 0& 0& 0& 1\\\hline
2-e&   1& 0& 1& 0& 0& 0& 0& 1& 1& 1& 0& 1& 1& 0& 1& 0\\\hline
3-1&   1& 0& 1& 0& 0& 1& 0& 0& 0& 1& 1& 1& 1& 0& 1& 0\\\hline
4-c&   0& 1& 0& 1& 1& 0& 1& 1& 1& 0& 0& 0& 0& 1& 0& 1\\\hline
5-7&   1& 1& 1& 0& 0& 1& 0& 1& 0& 1& 0& 1& 0& 0& 1& 0\\\hline
6-2&   0& 1& 0& 1& 0& 0& 1& 0& 1& 1& 1& 0& 0& 1& 0& 1\\\hline
7-3&   0& 1& 0& 1& 1& 1& 1& 0& 0& 0& 1& 0& 0& 1& 0& 1\\\hline
8-9&   0& 1& 1& 1& 1& 0& 1& 0& 1& 0& 1& 0& 0& 1& 0& 0\\\hline
9-4&   0& 1& 1& 1& 1& 0& 1& 0& 1& 0& 1& 0& 0& 1& 0& 0\\\hline
a-a&   1& 1& 1& 0& 0& 1& 0& 1& 0& 1& 0& 1& 0& 0& 1& 0\\\hline
b-8&   0& 1& 0& 0& 1& 0& 1& 0& 1& 0& 1& 0& 0& 1& 1& 1\\\hline
c-f&   0& 1& 0& 1& 0& 0& 1& 0& 1& 1& 1& 0& 0& 1& 0& 1\\\hline
d-0&   0& 1& 0& 1& 1& 0& 0& 0& 1& 0& 1& 1& 0& 1& 0& 1\\\hline
e-6&   1& 1& 0& 1& 1& 0& 1& 0& 1& 0& 1& 0& 0& 0& 0& 1\\\hline
f-5&   0& 1& 0& 0& 1& 0& 1& 0& 1& 0& 1& 0& 0& 1& 1& 1\\\hline\hline

$\displaystyle \sum T_0$ & 10&2& 10&6&6& 12&6& 12&4& 10&4& 10& 14&6& 10&6\\\hline
$\displaystyle \sum T_1$ & 6&14& 6&10&10& 4&10& 4&12& 6&12& 6& 2&10& 6&10\\\hline

\end{tabular}
\vspace{0.25cm}

The table generated for guesses of $k_3$.\\
\end{center}

From the table for $k_3$ we can observe that the probable candidates for $k_3$ are \{1,c\} as their $T_0$ and $T_1$ values are the most imbalanced.

\begin{center}

\begin{tabular}{*{17}{c|}}
\diagbox{p-c}{$k_0$} & 0 & 1 & 2 & 3 & 4 & 5 & 6 & 7 & 8 & 9 & a & b & c & d & e & f \\\hline
0-d&   0& 1& 1& 1& 0& 0& 1& 0& 1& 0& 1& 0& 0& 1& 0& 1\\\hline
1-b&   0& 1& 0& 0& 1& 1& 1& 0& 1& 0& 1& 0& 0& 1& 0& 1\\\hline
2-e&   0& 0& 1& 0& 0& 1& 1& 1& 0& 1& 0& 1& 1& 0& 1& 0\\\hline
3-1&   1& 1& 1& 0& 0& 1& 0& 0& 0& 1& 0& 1& 1& 0& 1& 0\\\hline
4-c&   1& 1& 0& 1& 1& 0& 0& 0& 1& 0& 1& 0& 0& 1& 0& 1\\\hline
5-7&   1& 1& 1& 0& 0& 1& 0& 0& 0& 1& 0& 1& 1& 0& 1& 0\\\hline
6-2&   0& 1& 1& 1& 0& 0& 1& 0& 1& 0& 1& 0& 0& 1& 0& 1\\\hline
7-3&   0& 1& 0& 0& 1& 1& 1& 0& 1& 0& 1& 0& 0& 1& 0& 1\\\hline
8-9&   0& 1& 0& 1& 1& 0& 1& 0& 1& 0& 0& 0& 1& 1& 0& 1\\\hline
9-4&   0& 1& 0& 1& 1& 0& 1& 0& 1& 0& 1& 1& 0& 0& 0& 1\\\hline
a-a&   1& 0& 1& 0& 0& 1& 0& 1& 1& 1& 0& 1& 1& 0& 0& 0\\\hline
b-8&   0& 1& 0& 1& 1& 0& 1& 0& 1& 1& 1& 0& 0& 1& 0& 0\\\hline
c-f&   0& 1& 0& 1& 1& 0& 1& 0& 0& 0& 1& 0& 0& 1& 1& 1\\\hline
d-0&   0& 1& 0& 1& 1& 0& 1& 0& 1& 1& 1& 0& 0& 1& 0& 0\\\hline
e-6&   0& 1& 0& 1& 1& 0& 1& 0& 1& 0& 0& 0& 1& 1& 0& 1\\\hline
f-5&   0& 1& 0& 1& 1& 0& 1& 0& 1& 0& 1& 1& 0& 0& 0& 1\\\hline\hline

$\displaystyle \sum T_0$ & 12&2& 10&6&6& 10&4& 14&4& 10&6& 10& 10&6& 12&6\\\hline
$\displaystyle \sum T_1$ & 4&14& 6&10&10& 6&12& 2&12& 6&10& 6& 6&10& 4&10\\\hline

\end{tabular}

\vspace{0.25cm}

The table generated for guesses of $k_0$.

\end{center}

From the table for $k_0$ we can observe that the probable candidates for $k_0$ are \{1,7\} as their $T_0$ and $T_1$ values are the most imbalanced. The guess of $k_0 = 6$, which is the exact value, does not have the most imbalanced sum as SNR is low for smaller bits.

To find $k_1$ and $k_2$, since we can verify $k_0$ and $k_3$ from the above candidates, the cipher will be reduced to Sypher00A. Then we can perform linear cryptanalysis on them using the characteristic $\alpha \xrightarrow{S} \beta = p$. We can reuse $d$ as the mask for both since it has the highest bias as seen in the LAT.
\end{large}


\section{State-Meant (AES)}

\begin{large}

Plaintext to work on: "ROHIT DAS". Padding scheme used: ANSIX9.23\\
The initial state is derived as follows (padding already in HEX):\\

\begin{center}

\begin{tabu}{*{4}{|[1.5pt]p{0.5cm}}|[1.5pt]}

\tabucline[1.5pt]{-}
\rowcolor[rgb]{0.96,0.37,0.37}R&T&S&00\\\tabucline[1.5pt]{-}
\rowcolor[rgb]{0.96,0.37,0.37}O& &00&00\\\tabucline[1.5pt]{-}
\rowcolor[rgb]{0.96,0.37,0.37}H&D&00&00\\\tabucline[1.5pt]{-}
\rowcolor[rgb]{0.96,0.37,0.37}I&A&00&07\\\tabucline[1.5pt]{-}

\end{tabu}
\large $\xRightarrow[\text{with ASCII}]{\text{Replacing}}$
\begin{tabu}{*{4}{|[1.5pt]p{0.5cm}}|[1.5pt]}

\tabucline[1.5pt]{-}
\rowcolor[rgb]{0.96,0.37,0.37}82&84&83&00\\\tabucline[1.5pt]{-}
\rowcolor[rgb]{0.96,0.37,0.37}79&32&00&00\\\tabucline[1.5pt]{-}
\rowcolor[rgb]{0.96,0.37,0.37}72&68&00&00\\\tabucline[1.5pt]{-}
\rowcolor[rgb]{0.96,0.37,0.37}73&65&00&07\\\tabucline[1.5pt]{-}

\end{tabu}
\large $\xRightarrow[\text{of ASCII}]{\text{HEX code}}$
\begin{tabu}{*{4}{|[1.5pt]p{0.5cm}}|[1.5pt]}

\tabucline[1.5pt]{-}
\rowcolor[rgb]{0.96,0.37,0.37}52&54&53&00\\\tabucline[1.5pt]{-}
\rowcolor[rgb]{0.96,0.37,0.37}4F&20&00&00\\\tabucline[1.5pt]{-}
\rowcolor[rgb]{0.96,0.37,0.37}48&44&00&00\\\tabucline[1.5pt]{-}
\rowcolor[rgb]{0.96,0.37,0.37}49&41&00&07\\\tabucline[1.5pt]{-}

\end{tabu}

\end{center}

\subsection{ShiftRows}

After applying ShiftRows, the state is as follows:

\begin{center}

\begin{tabu}{*{4}{|[1.5pt]p{0.5cm}}|[1.5pt]}

\tabucline[1.5pt]{-}
\tabucline[1.5pt]{-}
\cellcolor[rgb]{1,0.91,0}52
&\cellcolor[rgb]{0.12,0.87,0.13}54
&\cellcolor[rgb]{0.37,0.37,0.96}53
&\cellcolor[rgb]{0.96,0.37,0.37}00\\
%
\tabucline[1.5pt]{-}
\tabucline[1.5pt]{-}
\cellcolor[rgb]{1,0.91,0}4F
&\cellcolor[rgb]{0.12,0.87,0.13}20
&\cellcolor[rgb]{0.37,0.37,0.96}00
&\cellcolor[rgb]{0.96,0.37,0.37}00\\
%
\tabucline[1.5pt]{-}
\tabucline[1.5pt]{-}
\cellcolor[rgb]{1,0.91,0}48
&\cellcolor[rgb]{0.12,0.87,0.13}44
&\cellcolor[rgb]{0.37,0.37,0.96}00
&\cellcolor[rgb]{0.96,0.37,0.37}00\\
%
\tabucline[1.5pt]{-}
\tabucline[1.5pt]{-}
\cellcolor[rgb]{1,0.91,0}49
&\cellcolor[rgb]{0.12,0.87,0.13}41
&\cellcolor[rgb]{0.37,0.37,0.96}00
&\cellcolor[rgb]{0.96,0.37,0.37}07\\
\tabucline[1.5pt]{-}

\end{tabu}
\large $\xRightarrow[\text{by 0,1,2 and 3 places}]{\text{Shifting rows}}$
\begin{tabu}{*{4}{|[1.5pt]p{0.5cm}}|[1.5pt]}

\tabucline[1.5pt]{-}
\tabucline[1.5pt]{-}
\cellcolor[rgb]{1,0.91,0}52
&\cellcolor[rgb]{0.12,0.87,0.13}54
&\cellcolor[rgb]{0.37,0.37,0.96}53
&\cellcolor[rgb]{0.96,0.37,0.37}00\\
%
\tabucline[1.5pt]{-}
\tabucline[1.5pt]{-}
\cellcolor[rgb]{0.12,0.87,0.13}20
&\cellcolor[rgb]{0.37,0.37,0.96}00
&\cellcolor[rgb]{0.96,0.37,0.37}00
&\cellcolor[rgb]{1,0.91,0}4F\\
%
\tabucline[1.5pt]{-}
\tabucline[1.5pt]{-}
\cellcolor[rgb]{0.37,0.37,0.96}00
&\cellcolor[rgb]{0.96,0.37,0.37}00
&\cellcolor[rgb]{1,0.91,0}48
&\cellcolor[rgb]{0.12,0.87,0.13}44\\
%
\tabucline[1.5pt]{-}
\tabucline[1.5pt]{-}
\cellcolor[rgb]{0.96,0.37,0.37}07
&\cellcolor[rgb]{1,0.91,0}49
&\cellcolor[rgb]{0.12,0.87,0.13}41
&\cellcolor[rgb]{0.37,0.37,0.96}00\\
\tabucline[1.5pt]{-}

\end{tabu}

\end{center}

\subsection{SubBytes}
After applying SubBytes, the state is as follows:

\begin{center}

\begin{tabu}{*{4}{|[1.5pt]p{0.5cm}}|[1.5pt]}
\tabucline[1.5pt]{-}
\rowcolor[rgb]{0.96,0.37,0.37}52&54&53&00\\\tabucline[1.5pt]{-}
\rowcolor[rgb]{0.96,0.37,0.37}20&00&00&4F\\\tabucline[1.5pt]{-}
\rowcolor[rgb]{0.96,0.37,0.37}00&00&48&44\\\tabucline[1.5pt]{-}
\rowcolor[rgb]{0.96,0.37,0.37}07&49&41&00\\\tabucline[1.5pt]{-}
\end{tabu}
\large $\xRightarrow[\text{bytes from S-Box}]{\text{Substituting}}$
\begin{tabu}{*{4}{|[1.5pt]p{0.5cm}}|[1.5pt]}
\tabucline[1.5pt]{-}
\rowcolor[rgb]{0.37,0.37,0.96}00&20&ED&63\\\tabucline[1.5pt]{-}
\rowcolor[rgb]{0.37,0.37,0.96}B7&63&63&63\\\tabucline[1.5pt]{-}
\rowcolor[rgb]{0.37,0.37,0.96}63&63&52&1B\\\tabucline[1.5pt]{-}
\rowcolor[rgb]{0.37,0.37,0.96}C5&3B&83&63\\\tabucline[1.5pt]{-}
\end{tabu}

\end{center}

\end{large}

\end{document}