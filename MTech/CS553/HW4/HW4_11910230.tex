\documentclass[12pt]{article}

\usepackage[a4paper, left = 1.5cm, right = 1.5cm, top = 1.5cm, bottom = 1.5cm]{geometry}

\usepackage{amsmath}
\usepackage{amssymb}

\usepackage{enumitem}

\usepackage{graphicx}

\usepackage{karnaugh-map}

\usepackage{listings}
\usepackage{xcolor}
\lstset{
	language = Python,
	backgroundcolor=\color{black!5}, % set backgroundcolor
}

\renewcommand{\baselinestretch}{1.5}

\renewcommand\thesection{\arabic{section}.}

\usepackage{tocloft}
\cftsetindents{section}{0em}{2em}
\cftsetindents{subsection}{0em}{2em}
\renewcommand\cfttoctitlefont{\hfill\Large\bfseries}
\renewcommand\cftaftertoctitle{\hfill\mbox{}}
\setcounter{tocdepth}{2}

\begin{document}
\begin{titlepage}
\centering
\vspace*{\fill}
\huge CS553: Cryptography\\
\LARGE Assignment 4: Solutions\\
\Large Rohit Das (11910230)\\\vspace{0.8cm}
\today
\vspace*{\fill}
\end{titlepage}

\section{Fiestal and SPN}
\begin{large}
\subsection{Fiestal Ciphers}
\begin{itemize}[itemsep=-1ex, topsep=-1ex]
\item \textbf{Blowfish:}
	\begin{itemize}[itemsep=-2ex, topsep=-2ex]
	\item Block Size: 64 bits
	\item Key Size: 32 - 448 bits
	\item Susceptible to $2^{nd}$-order differential attack.
	\end{itemize}
	
\item \textbf{Data Encryption Standard (DES):}
	\begin{itemize}[itemsep=-2ex, topsep=-2ex]
	\item Block Size: 64 bits
	\item Key Size: 56 bits (+8 parity bits)
	\item Considered insecure because of feasibility of brute-force attacks.
	\end{itemize}
	
\item \textbf{Rivest Cipher (RC5):}
	\begin{itemize}[itemsep=-2ex, topsep=-2ex]
	\item Block Size: 32/64 (suggested)/128 bits
	\item Key Size: 0 - 2040 bits (128 bits suggested)
	\item Susceptible to differential attacks using $2^{44}$ plaintexts.
	\end{itemize}
	
\item \textbf{Information Concealment Engine (ICE):}
	\begin{itemize}[itemsep=-2ex, topsep=-2ex]
	\item Block Size: 64 bits
	\item Key Size: 64 bits
	\item Differential attacks can break 15 of 16 rounds with complexity $2^{56}$.
	\end{itemize}
	
\item \textbf{KASUMI:}
	\begin{itemize}[itemsep=-2ex, topsep=-2ex]
	\item Block Size: 64 bits
	\item Key Size: 128 bits
	\end{itemize}

\end{itemize}

\subsection{Substitution-Permutation Network (SPN)}
\begin{itemize}[itemsep=-1ex, topsep=-1ex]

\item \textbf{Advanced Encryption Standard (AES):}
	\begin{itemize}[itemsep=-2ex, topsep=-2ex]
	\item Block Size: 128 bits
	\item Key Size: 128/192/256 bits
	\item For AES-128, key can be recovered with complexity $2^{126.1}$ (biclique attack).
	\end{itemize}
	
\item \textbf{3-Way:}
	\begin{itemize}[itemsep=-2ex, topsep=-2ex]
	\item Block Size: 96 bits
	\item Key Size: 96 bits
	\item Vulnerable to related key cryptanalysis.
	\end{itemize}
	
\item \textbf{Kuznyechik:}
	\begin{itemize}[itemsep=-2ex, topsep=-2ex]
	\item Block Size: 128 bits
	\item Key Size: 256 bits
	\item Vulnerable to meet-in-the-middle attack on 5 rounds.
	\end{itemize}
	
\item \textbf{SAFER K-64 (Safer And Faster Encryption Routine):}
	\begin{itemize}[itemsep=-2ex, topsep=-2ex]
	\item Block Size: 64 bits
	\item Key Size: 64 bits
	\end{itemize}
	
\item \textbf{Square:}
	\begin{itemize}[itemsep=-2ex, topsep=-2ex]
	\item Block Size: 96 bits
	\item Key Size: 96 bits
	\item Precursor to AES.
	\end{itemize}
	
\end{itemize}

\end{large}

\section{Random SBox (4-bit)}
\begin{large}
Python code (Python 3): Random\textunderscore s-box\textunderscore gen.py
\lstinputlisting[language = Python, showstringspaces=false]{./python_code/Random_s-box_gen.py}

The S-Box to be used subsequently is as follows:
\begin{center}

\begin{tabular}{|c|c|c|c|c|c|c|c|c|c|c|c|c|c|c|c|c|}
\hline
$x$ & 0 & 1 & 2 & 3 & 4 & 5 & 6 & 7 & 8 & 9 & a & b & c & d & e & f\\\hline
$S(x)$ & 5 & 4 & d & 1 & 2 & f & 6 & 0 & 8 & c & b & 9 & 7 & e & a & 3\\\hline
\end{tabular}

\end{center}
\end{large}
\vspace{0.5cm}

\section{Differential Distribution Table (DDT)}
\begin{large}

Python code (Python 3): DDT\textunderscore s-box.py

\lstinputlisting[language = Python, showstringspaces=false]{./python_code/DDT_s-box.py}

The generated DDT is as follows:
\begin{center}

\begin{tabular}{c||p{0.5cm}p{0.5cm}p{0.5cm}p{0.5cm}p{0.5cm}p{0.5cm}p{0.5cm}p{0.5cm}p{0.5cm}p{0.5cm}p{0.5cm}p{0.5cm}p{0.5cm}p{0.5cm}p{0.5cm}p{0.5cm}}

in\textbackslash out & 0 & 1 & 2 & 3 & 4 & 5 & 6 & 7 & 8 & 9 & a & b & c & d & e & f\\\hline
0& 16& -& -& -& -& -& -& -& -& -& -& -& -& -& -& - \\
1&  -& 2& 2& -& 2& -& 2& -& -& 4& -& -& 2& 2& -& - \\
2&  -& -& -& 2& 2& 4& -& -& 2& -& -& -& -& 4& -& 2 \\
3&  -& 2& 2& -& 6& -& -& 2& -& 4& -& -& -& -& -& - \\
4&  -& 4& 2& -& -& -& -& 2& -& -& 2& 4& -& -& -& 2 \\
5&  -& -& -& 2& -& -& 4& 2& 2& -& 2& 2& -& 2& -& - \\
6&  -& -& 2& 2& 2& -& -& 2& -& -& -& -& 2& -& 2& 4 \\
7&  -& -& 4& 2& -& 4& 2& -& -& -& -& 2& -& -& 2& - \\
8&  -& 2& -& 2& -& 2& 2& -& 4& -& -& -& 2& 2& -& - \\
9&  -& -& -& -& 2& 2& -& -& 2& 2& 4& -& 4& -& -& - \\
a&  -& 2& -& -& -& 2& -& -& 2& -& -& -& 2& 4& 4& - \\
b&  -& 4& -& -& -& 2& -& 2& 2& 2& -& -& 2& -& -& 2 \\
c&  -& -& 4& 2& -& -& -& 2& -& 2& 4& -& -& 2& -& - \\
d&  -& -& -& 2& -& -& -& 2& -& -& -& 6& -& -& 4& 2 \\
e&  -& -& -& -& -& -& 2& 2& -& 2& 2& -& 2& -& 2& 4 \\
f&  -& -& -& 2& 2& -& 4& -& 2& -& 2& 2& -& -& 2& - \\
\end{tabular}

\end{center}

Actual output(file): output.txt

The maximum differential probability of this S-Box is $\dfrac{6}{16}$ for the (input,output) difference transactions (3,4) and (d,b).

\end{large}

\section{SBox as a Boolean Function}
\begin{large}

The Boolean table for the above S-Box is as follows:\\

\begin{center}

\begin{tabular}{|c|cccc|p{1cm}|p{1cm}|p{1cm}|p{1cm}|c|}
\hline
$x$ & $x_0$ & $x_1$ & $x_2$ & $x_3$ & $y_0$ & $y_1$ & $y_2$ & $y_3$ & $y = S[x]$\\\hline
0 & 0 & 0 & 0 & 0 & 0 & 1 & 0 & 1 & 5\\\hline
1 & 0 & 0 & 0 & 1 & 0 & 1 & 0 & 0 & 4\\\hline
2 & 0 & 0 & 1 & 0 & 1 & 1 & 0 & 1 & d\\\hline
3 & 0 & 0 & 1 & 1 & 0 & 0 & 0 & 1 & 1\\\hline
4 & 0 & 1 & 0 & 0 & 0 & 0 & 1 & 0 & 2\\\hline
5 & 0 & 1 & 0 & 1 & 1 & 1 & 1 & 1 & f\\\hline
6 & 0 & 1 & 1 & 0 & 0 & 1 & 1 & 0 & 6\\\hline
7 & 0 & 1 & 1 & 1 & 0 & 0 & 0 & 0 & 0\\\hline
8 & 1 & 0 & 0 & 0 & 1 & 0 & 0 & 0 & 8\\\hline
9 & 1 & 0 & 0 & 1 & 1 & 1 & 0 & 0 & c\\\hline
a & 1 & 0 & 1 & 0 & 1 & 0 & 1 & 1 & b\\\hline
b & 1 & 0 & 1 & 1 & 1 & 0 & 0 & 1 & 9\\\hline
c & 1 & 1 & 0 & 0 & 0 & 1 & 1 & 1 & 7\\\hline
d & 1 & 1 & 0 & 1 & 1 & 1 & 1 & 0 & e\\\hline
e & 1 & 1 & 1 & 0 & 1 & 0 & 1 & 0 & a\\\hline
f & 1 & 1 & 1 & 1 & 0 & 0 & 1 & 1 & 3\\\hline

\end{tabular}

\end{center}

\newpage
The 4 K-maps for the variables are as follows:

\begin{karnaugh-map}[4][4][1][$X_2X_3$][$X_0X_1$]

\minterms{2,5,8,9,10,11,13,14}
\autoterms[0]
\implicant{8}{10}
\implicant{5}{13}
\implicant{14}{10}
\implicantedge{2}{2}{10}{10}

\end{karnaugh-map}\hspace{2cm}
\begin{karnaugh-map}[4][4][1][$X_2X_3$][$X_0X_1$]

\minterms{0,1,2,5,6,9,12,13}
\autoterms[0]
\implicant{1}{9}
\implicant{12}{13}
\implicant{0}{1}
\implicant{2}{6}

\end{karnaugh-map}

\vspace{-1cm}
\hspace{0.5cm}$y_0 = \sum (2,5,8,9,10,11,13,14)$ \hspace{3.25cm}$y_1 = \sum (0,1,2,5,6,9,12,13)$
\vspace{1cm}

\begin{karnaugh-map}[4][4][1][$X_2X_3$][$X_0X_1$]

\minterms{4,5,6,10,12,13,14,15}
\autoterms[0]
\implicant{12}{14}
\implicant{4}{5}
\implicant{6}{14}
\implicant{14}{10}

\end{karnaugh-map}\hspace{2cm}
\begin{karnaugh-map}[4][4][1][$X_2X_3$][$X_0X_1$]

\minterms{0,2,3,5,10,11,12,15}
\autoterms[0]
\implicantedge{3}{2}{11}{10}
\implicantedge{0}{0}{2}{2}
\implicant{15}{11}
\implicant{5}{5}
\implicant{12}{12}

\end{karnaugh-map}

\vspace{-1cm}
\hspace{0.25cm}$y_2 = \sum (4,5,6,10,12,13,14,15)$ \hspace{2.75cm}$y_3 = \sum (0,2,3,5,10,11,12,15)$
\vspace{1cm}

The formulas for the variables are as follows:\\
$y_0 = \sum (2,5,8,9,10,11,13,14) = x_0x_1^{'} + x_1x_2^{'}x_3 + x_0x_2x_3^{'} + x_1^{'}x_2x_3^{'}$\\
$y_1 = \sum (0,1,2,5,6,9,12,13) = x_2^{'}x_3 + x_0^{'}x_1^{'}x_2^{'} + x_0^{'}x_2x_3^{'} + x_0x_1x_2^{'}$\\
$y_2 = \sum (4,5,6,10,12,13,14,15) = x_0x_1 + x_0^{'}x_1x_2^{'} + x_1x_2x_3^{'} + x_0x_2x_3^{'}$\\
$y_3 = \sum (0,2,3,5,10,11,12,15) = x_1^{'}x_2 + x_0^{'}x_1^{'}x_3^{'} + x_0^{'}x_2x_3 + x_0^{'}x_1x_2^{'}x_3 + x_0x_1x_2^{'}x_3^{'}$\\

\end{large}
\end{document}